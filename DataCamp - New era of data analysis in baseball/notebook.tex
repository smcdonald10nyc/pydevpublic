
% Default to the notebook output style

    


% Inherit from the specified cell style.




    
\documentclass[11pt]{article}

    
    
    \usepackage[T1]{fontenc}
    % Nicer default font (+ math font) than Computer Modern for most use cases
    \usepackage{mathpazo}

    % Basic figure setup, for now with no caption control since it's done
    % automatically by Pandoc (which extracts ![](path) syntax from Markdown).
    \usepackage{graphicx}
    % We will generate all images so they have a width \maxwidth. This means
    % that they will get their normal width if they fit onto the page, but
    % are scaled down if they would overflow the margins.
    \makeatletter
    \def\maxwidth{\ifdim\Gin@nat@width>\linewidth\linewidth
    \else\Gin@nat@width\fi}
    \makeatother
    \let\Oldincludegraphics\includegraphics
    % Set max figure width to be 80% of text width, for now hardcoded.
    \renewcommand{\includegraphics}[1]{\Oldincludegraphics[width=.8\maxwidth]{#1}}
    % Ensure that by default, figures have no caption (until we provide a
    % proper Figure object with a Caption API and a way to capture that
    % in the conversion process - todo).
    \usepackage{caption}
    \DeclareCaptionLabelFormat{nolabel}{}
    \captionsetup{labelformat=nolabel}

    \usepackage{adjustbox} % Used to constrain images to a maximum size 
    \usepackage{xcolor} % Allow colors to be defined
    \usepackage{enumerate} % Needed for markdown enumerations to work
    \usepackage{geometry} % Used to adjust the document margins
    \usepackage{amsmath} % Equations
    \usepackage{amssymb} % Equations
    \usepackage{textcomp} % defines textquotesingle
    % Hack from http://tex.stackexchange.com/a/47451/13684:
    \AtBeginDocument{%
        \def\PYZsq{\textquotesingle}% Upright quotes in Pygmentized code
    }
    \usepackage{upquote} % Upright quotes for verbatim code
    \usepackage{eurosym} % defines \euro
    \usepackage[mathletters]{ucs} % Extended unicode (utf-8) support
    \usepackage[utf8x]{inputenc} % Allow utf-8 characters in the tex document
    \usepackage{fancyvrb} % verbatim replacement that allows latex
    \usepackage{grffile} % extends the file name processing of package graphics 
                         % to support a larger range 
    % The hyperref package gives us a pdf with properly built
    % internal navigation ('pdf bookmarks' for the table of contents,
    % internal cross-reference links, web links for URLs, etc.)
    \usepackage{hyperref}
    \usepackage{longtable} % longtable support required by pandoc >1.10
    \usepackage{booktabs}  % table support for pandoc > 1.12.2
    \usepackage[inline]{enumitem} % IRkernel/repr support (it uses the enumerate* environment)
    \usepackage[normalem]{ulem} % ulem is needed to support strikethroughs (\sout)
                                % normalem makes italics be italics, not underlines
    

    
    
    % Colors for the hyperref package
    \definecolor{urlcolor}{rgb}{0,.145,.698}
    \definecolor{linkcolor}{rgb}{.71,0.21,0.01}
    \definecolor{citecolor}{rgb}{.12,.54,.11}

    % ANSI colors
    \definecolor{ansi-black}{HTML}{3E424D}
    \definecolor{ansi-black-intense}{HTML}{282C36}
    \definecolor{ansi-red}{HTML}{E75C58}
    \definecolor{ansi-red-intense}{HTML}{B22B31}
    \definecolor{ansi-green}{HTML}{00A250}
    \definecolor{ansi-green-intense}{HTML}{007427}
    \definecolor{ansi-yellow}{HTML}{DDB62B}
    \definecolor{ansi-yellow-intense}{HTML}{B27D12}
    \definecolor{ansi-blue}{HTML}{208FFB}
    \definecolor{ansi-blue-intense}{HTML}{0065CA}
    \definecolor{ansi-magenta}{HTML}{D160C4}
    \definecolor{ansi-magenta-intense}{HTML}{A03196}
    \definecolor{ansi-cyan}{HTML}{60C6C8}
    \definecolor{ansi-cyan-intense}{HTML}{258F8F}
    \definecolor{ansi-white}{HTML}{C5C1B4}
    \definecolor{ansi-white-intense}{HTML}{A1A6B2}

    % commands and environments needed by pandoc snippets
    % extracted from the output of `pandoc -s`
    \providecommand{\tightlist}{%
      \setlength{\itemsep}{0pt}\setlength{\parskip}{0pt}}
    \DefineVerbatimEnvironment{Highlighting}{Verbatim}{commandchars=\\\{\}}
    % Add ',fontsize=\small' for more characters per line
    \newenvironment{Shaded}{}{}
    \newcommand{\KeywordTok}[1]{\textcolor[rgb]{0.00,0.44,0.13}{\textbf{{#1}}}}
    \newcommand{\DataTypeTok}[1]{\textcolor[rgb]{0.56,0.13,0.00}{{#1}}}
    \newcommand{\DecValTok}[1]{\textcolor[rgb]{0.25,0.63,0.44}{{#1}}}
    \newcommand{\BaseNTok}[1]{\textcolor[rgb]{0.25,0.63,0.44}{{#1}}}
    \newcommand{\FloatTok}[1]{\textcolor[rgb]{0.25,0.63,0.44}{{#1}}}
    \newcommand{\CharTok}[1]{\textcolor[rgb]{0.25,0.44,0.63}{{#1}}}
    \newcommand{\StringTok}[1]{\textcolor[rgb]{0.25,0.44,0.63}{{#1}}}
    \newcommand{\CommentTok}[1]{\textcolor[rgb]{0.38,0.63,0.69}{\textit{{#1}}}}
    \newcommand{\OtherTok}[1]{\textcolor[rgb]{0.00,0.44,0.13}{{#1}}}
    \newcommand{\AlertTok}[1]{\textcolor[rgb]{1.00,0.00,0.00}{\textbf{{#1}}}}
    \newcommand{\FunctionTok}[1]{\textcolor[rgb]{0.02,0.16,0.49}{{#1}}}
    \newcommand{\RegionMarkerTok}[1]{{#1}}
    \newcommand{\ErrorTok}[1]{\textcolor[rgb]{1.00,0.00,0.00}{\textbf{{#1}}}}
    \newcommand{\NormalTok}[1]{{#1}}
    
    % Additional commands for more recent versions of Pandoc
    \newcommand{\ConstantTok}[1]{\textcolor[rgb]{0.53,0.00,0.00}{{#1}}}
    \newcommand{\SpecialCharTok}[1]{\textcolor[rgb]{0.25,0.44,0.63}{{#1}}}
    \newcommand{\VerbatimStringTok}[1]{\textcolor[rgb]{0.25,0.44,0.63}{{#1}}}
    \newcommand{\SpecialStringTok}[1]{\textcolor[rgb]{0.73,0.40,0.53}{{#1}}}
    \newcommand{\ImportTok}[1]{{#1}}
    \newcommand{\DocumentationTok}[1]{\textcolor[rgb]{0.73,0.13,0.13}{\textit{{#1}}}}
    \newcommand{\AnnotationTok}[1]{\textcolor[rgb]{0.38,0.63,0.69}{\textbf{\textit{{#1}}}}}
    \newcommand{\CommentVarTok}[1]{\textcolor[rgb]{0.38,0.63,0.69}{\textbf{\textit{{#1}}}}}
    \newcommand{\VariableTok}[1]{\textcolor[rgb]{0.10,0.09,0.49}{{#1}}}
    \newcommand{\ControlFlowTok}[1]{\textcolor[rgb]{0.00,0.44,0.13}{\textbf{{#1}}}}
    \newcommand{\OperatorTok}[1]{\textcolor[rgb]{0.40,0.40,0.40}{{#1}}}
    \newcommand{\BuiltInTok}[1]{{#1}}
    \newcommand{\ExtensionTok}[1]{{#1}}
    \newcommand{\PreprocessorTok}[1]{\textcolor[rgb]{0.74,0.48,0.00}{{#1}}}
    \newcommand{\AttributeTok}[1]{\textcolor[rgb]{0.49,0.56,0.16}{{#1}}}
    \newcommand{\InformationTok}[1]{\textcolor[rgb]{0.38,0.63,0.69}{\textbf{\textit{{#1}}}}}
    \newcommand{\WarningTok}[1]{\textcolor[rgb]{0.38,0.63,0.69}{\textbf{\textit{{#1}}}}}
    
    
    % Define a nice break command that doesn't care if a line doesn't already
    % exist.
    \def\br{\hspace*{\fill} \\* }
    % Math Jax compatability definitions
    \def\gt{>}
    \def\lt{<}
    % Document parameters
    \title{notebook}
    
    
    

    % Pygments definitions
    
\makeatletter
\def\PY@reset{\let\PY@it=\relax \let\PY@bf=\relax%
    \let\PY@ul=\relax \let\PY@tc=\relax%
    \let\PY@bc=\relax \let\PY@ff=\relax}
\def\PY@tok#1{\csname PY@tok@#1\endcsname}
\def\PY@toks#1+{\ifx\relax#1\empty\else%
    \PY@tok{#1}\expandafter\PY@toks\fi}
\def\PY@do#1{\PY@bc{\PY@tc{\PY@ul{%
    \PY@it{\PY@bf{\PY@ff{#1}}}}}}}
\def\PY#1#2{\PY@reset\PY@toks#1+\relax+\PY@do{#2}}

\expandafter\def\csname PY@tok@w\endcsname{\def\PY@tc##1{\textcolor[rgb]{0.73,0.73,0.73}{##1}}}
\expandafter\def\csname PY@tok@c\endcsname{\let\PY@it=\textit\def\PY@tc##1{\textcolor[rgb]{0.25,0.50,0.50}{##1}}}
\expandafter\def\csname PY@tok@cp\endcsname{\def\PY@tc##1{\textcolor[rgb]{0.74,0.48,0.00}{##1}}}
\expandafter\def\csname PY@tok@k\endcsname{\let\PY@bf=\textbf\def\PY@tc##1{\textcolor[rgb]{0.00,0.50,0.00}{##1}}}
\expandafter\def\csname PY@tok@kp\endcsname{\def\PY@tc##1{\textcolor[rgb]{0.00,0.50,0.00}{##1}}}
\expandafter\def\csname PY@tok@kt\endcsname{\def\PY@tc##1{\textcolor[rgb]{0.69,0.00,0.25}{##1}}}
\expandafter\def\csname PY@tok@o\endcsname{\def\PY@tc##1{\textcolor[rgb]{0.40,0.40,0.40}{##1}}}
\expandafter\def\csname PY@tok@ow\endcsname{\let\PY@bf=\textbf\def\PY@tc##1{\textcolor[rgb]{0.67,0.13,1.00}{##1}}}
\expandafter\def\csname PY@tok@nb\endcsname{\def\PY@tc##1{\textcolor[rgb]{0.00,0.50,0.00}{##1}}}
\expandafter\def\csname PY@tok@nf\endcsname{\def\PY@tc##1{\textcolor[rgb]{0.00,0.00,1.00}{##1}}}
\expandafter\def\csname PY@tok@nc\endcsname{\let\PY@bf=\textbf\def\PY@tc##1{\textcolor[rgb]{0.00,0.00,1.00}{##1}}}
\expandafter\def\csname PY@tok@nn\endcsname{\let\PY@bf=\textbf\def\PY@tc##1{\textcolor[rgb]{0.00,0.00,1.00}{##1}}}
\expandafter\def\csname PY@tok@ne\endcsname{\let\PY@bf=\textbf\def\PY@tc##1{\textcolor[rgb]{0.82,0.25,0.23}{##1}}}
\expandafter\def\csname PY@tok@nv\endcsname{\def\PY@tc##1{\textcolor[rgb]{0.10,0.09,0.49}{##1}}}
\expandafter\def\csname PY@tok@no\endcsname{\def\PY@tc##1{\textcolor[rgb]{0.53,0.00,0.00}{##1}}}
\expandafter\def\csname PY@tok@nl\endcsname{\def\PY@tc##1{\textcolor[rgb]{0.63,0.63,0.00}{##1}}}
\expandafter\def\csname PY@tok@ni\endcsname{\let\PY@bf=\textbf\def\PY@tc##1{\textcolor[rgb]{0.60,0.60,0.60}{##1}}}
\expandafter\def\csname PY@tok@na\endcsname{\def\PY@tc##1{\textcolor[rgb]{0.49,0.56,0.16}{##1}}}
\expandafter\def\csname PY@tok@nt\endcsname{\let\PY@bf=\textbf\def\PY@tc##1{\textcolor[rgb]{0.00,0.50,0.00}{##1}}}
\expandafter\def\csname PY@tok@nd\endcsname{\def\PY@tc##1{\textcolor[rgb]{0.67,0.13,1.00}{##1}}}
\expandafter\def\csname PY@tok@s\endcsname{\def\PY@tc##1{\textcolor[rgb]{0.73,0.13,0.13}{##1}}}
\expandafter\def\csname PY@tok@sd\endcsname{\let\PY@it=\textit\def\PY@tc##1{\textcolor[rgb]{0.73,0.13,0.13}{##1}}}
\expandafter\def\csname PY@tok@si\endcsname{\let\PY@bf=\textbf\def\PY@tc##1{\textcolor[rgb]{0.73,0.40,0.53}{##1}}}
\expandafter\def\csname PY@tok@se\endcsname{\let\PY@bf=\textbf\def\PY@tc##1{\textcolor[rgb]{0.73,0.40,0.13}{##1}}}
\expandafter\def\csname PY@tok@sr\endcsname{\def\PY@tc##1{\textcolor[rgb]{0.73,0.40,0.53}{##1}}}
\expandafter\def\csname PY@tok@ss\endcsname{\def\PY@tc##1{\textcolor[rgb]{0.10,0.09,0.49}{##1}}}
\expandafter\def\csname PY@tok@sx\endcsname{\def\PY@tc##1{\textcolor[rgb]{0.00,0.50,0.00}{##1}}}
\expandafter\def\csname PY@tok@m\endcsname{\def\PY@tc##1{\textcolor[rgb]{0.40,0.40,0.40}{##1}}}
\expandafter\def\csname PY@tok@gh\endcsname{\let\PY@bf=\textbf\def\PY@tc##1{\textcolor[rgb]{0.00,0.00,0.50}{##1}}}
\expandafter\def\csname PY@tok@gu\endcsname{\let\PY@bf=\textbf\def\PY@tc##1{\textcolor[rgb]{0.50,0.00,0.50}{##1}}}
\expandafter\def\csname PY@tok@gd\endcsname{\def\PY@tc##1{\textcolor[rgb]{0.63,0.00,0.00}{##1}}}
\expandafter\def\csname PY@tok@gi\endcsname{\def\PY@tc##1{\textcolor[rgb]{0.00,0.63,0.00}{##1}}}
\expandafter\def\csname PY@tok@gr\endcsname{\def\PY@tc##1{\textcolor[rgb]{1.00,0.00,0.00}{##1}}}
\expandafter\def\csname PY@tok@ge\endcsname{\let\PY@it=\textit}
\expandafter\def\csname PY@tok@gs\endcsname{\let\PY@bf=\textbf}
\expandafter\def\csname PY@tok@gp\endcsname{\let\PY@bf=\textbf\def\PY@tc##1{\textcolor[rgb]{0.00,0.00,0.50}{##1}}}
\expandafter\def\csname PY@tok@go\endcsname{\def\PY@tc##1{\textcolor[rgb]{0.53,0.53,0.53}{##1}}}
\expandafter\def\csname PY@tok@gt\endcsname{\def\PY@tc##1{\textcolor[rgb]{0.00,0.27,0.87}{##1}}}
\expandafter\def\csname PY@tok@err\endcsname{\def\PY@bc##1{\setlength{\fboxsep}{0pt}\fcolorbox[rgb]{1.00,0.00,0.00}{1,1,1}{\strut ##1}}}
\expandafter\def\csname PY@tok@kc\endcsname{\let\PY@bf=\textbf\def\PY@tc##1{\textcolor[rgb]{0.00,0.50,0.00}{##1}}}
\expandafter\def\csname PY@tok@kd\endcsname{\let\PY@bf=\textbf\def\PY@tc##1{\textcolor[rgb]{0.00,0.50,0.00}{##1}}}
\expandafter\def\csname PY@tok@kn\endcsname{\let\PY@bf=\textbf\def\PY@tc##1{\textcolor[rgb]{0.00,0.50,0.00}{##1}}}
\expandafter\def\csname PY@tok@kr\endcsname{\let\PY@bf=\textbf\def\PY@tc##1{\textcolor[rgb]{0.00,0.50,0.00}{##1}}}
\expandafter\def\csname PY@tok@bp\endcsname{\def\PY@tc##1{\textcolor[rgb]{0.00,0.50,0.00}{##1}}}
\expandafter\def\csname PY@tok@fm\endcsname{\def\PY@tc##1{\textcolor[rgb]{0.00,0.00,1.00}{##1}}}
\expandafter\def\csname PY@tok@vc\endcsname{\def\PY@tc##1{\textcolor[rgb]{0.10,0.09,0.49}{##1}}}
\expandafter\def\csname PY@tok@vg\endcsname{\def\PY@tc##1{\textcolor[rgb]{0.10,0.09,0.49}{##1}}}
\expandafter\def\csname PY@tok@vi\endcsname{\def\PY@tc##1{\textcolor[rgb]{0.10,0.09,0.49}{##1}}}
\expandafter\def\csname PY@tok@vm\endcsname{\def\PY@tc##1{\textcolor[rgb]{0.10,0.09,0.49}{##1}}}
\expandafter\def\csname PY@tok@sa\endcsname{\def\PY@tc##1{\textcolor[rgb]{0.73,0.13,0.13}{##1}}}
\expandafter\def\csname PY@tok@sb\endcsname{\def\PY@tc##1{\textcolor[rgb]{0.73,0.13,0.13}{##1}}}
\expandafter\def\csname PY@tok@sc\endcsname{\def\PY@tc##1{\textcolor[rgb]{0.73,0.13,0.13}{##1}}}
\expandafter\def\csname PY@tok@dl\endcsname{\def\PY@tc##1{\textcolor[rgb]{0.73,0.13,0.13}{##1}}}
\expandafter\def\csname PY@tok@s2\endcsname{\def\PY@tc##1{\textcolor[rgb]{0.73,0.13,0.13}{##1}}}
\expandafter\def\csname PY@tok@sh\endcsname{\def\PY@tc##1{\textcolor[rgb]{0.73,0.13,0.13}{##1}}}
\expandafter\def\csname PY@tok@s1\endcsname{\def\PY@tc##1{\textcolor[rgb]{0.73,0.13,0.13}{##1}}}
\expandafter\def\csname PY@tok@mb\endcsname{\def\PY@tc##1{\textcolor[rgb]{0.40,0.40,0.40}{##1}}}
\expandafter\def\csname PY@tok@mf\endcsname{\def\PY@tc##1{\textcolor[rgb]{0.40,0.40,0.40}{##1}}}
\expandafter\def\csname PY@tok@mh\endcsname{\def\PY@tc##1{\textcolor[rgb]{0.40,0.40,0.40}{##1}}}
\expandafter\def\csname PY@tok@mi\endcsname{\def\PY@tc##1{\textcolor[rgb]{0.40,0.40,0.40}{##1}}}
\expandafter\def\csname PY@tok@il\endcsname{\def\PY@tc##1{\textcolor[rgb]{0.40,0.40,0.40}{##1}}}
\expandafter\def\csname PY@tok@mo\endcsname{\def\PY@tc##1{\textcolor[rgb]{0.40,0.40,0.40}{##1}}}
\expandafter\def\csname PY@tok@ch\endcsname{\let\PY@it=\textit\def\PY@tc##1{\textcolor[rgb]{0.25,0.50,0.50}{##1}}}
\expandafter\def\csname PY@tok@cm\endcsname{\let\PY@it=\textit\def\PY@tc##1{\textcolor[rgb]{0.25,0.50,0.50}{##1}}}
\expandafter\def\csname PY@tok@cpf\endcsname{\let\PY@it=\textit\def\PY@tc##1{\textcolor[rgb]{0.25,0.50,0.50}{##1}}}
\expandafter\def\csname PY@tok@c1\endcsname{\let\PY@it=\textit\def\PY@tc##1{\textcolor[rgb]{0.25,0.50,0.50}{##1}}}
\expandafter\def\csname PY@tok@cs\endcsname{\let\PY@it=\textit\def\PY@tc##1{\textcolor[rgb]{0.25,0.50,0.50}{##1}}}

\def\PYZbs{\char`\\}
\def\PYZus{\char`\_}
\def\PYZob{\char`\{}
\def\PYZcb{\char`\}}
\def\PYZca{\char`\^}
\def\PYZam{\char`\&}
\def\PYZlt{\char`\<}
\def\PYZgt{\char`\>}
\def\PYZsh{\char`\#}
\def\PYZpc{\char`\%}
\def\PYZdl{\char`\$}
\def\PYZhy{\char`\-}
\def\PYZsq{\char`\'}
\def\PYZdq{\char`\"}
\def\PYZti{\char`\~}
% for compatibility with earlier versions
\def\PYZat{@}
\def\PYZlb{[}
\def\PYZrb{]}
\makeatother


    % Exact colors from NB
    \definecolor{incolor}{rgb}{0.0, 0.0, 0.5}
    \definecolor{outcolor}{rgb}{0.545, 0.0, 0.0}



    
    % Prevent overflowing lines due to hard-to-break entities
    \sloppy 
    % Setup hyperref package
    \hypersetup{
      breaklinks=true,  % so long urls are correctly broken across lines
      colorlinks=true,
      urlcolor=urlcolor,
      linkcolor=linkcolor,
      citecolor=citecolor,
      }
    % Slightly bigger margins than the latex defaults
    
    \geometry{verbose,tmargin=1in,bmargin=1in,lmargin=1in,rmargin=1in}
    
    

    \begin{document}
    
    
    \maketitle
    
    

    
    \hypertarget{the-statcast-revolution}{%
\subsection{1. The Statcast revolution}\label{the-statcast-revolution}}

This is Aaron Judge. Judge is one of the physically largest players in
Major League Baseball standing 6 feet 7 inches (2.01 m) tall and
weighing 282 pounds (128 kg). He also hit the hardest home run ever
recorded. How do we know this? Statcast.

Statcast is a state-of-the-art tracking system that uses high-resolution
cameras and radar equipment to measure the precise location and movement
of baseballs and baseball players. Introduced in 2015 to all 30 major
league ballparks, Statcast data is revolutionizing the game. Teams are
engaging in an ``arms race'' of data analysis, hiring analysts left and
right in an attempt to gain an edge over their competition. This video
describing the system is incredible.

In this notebook, we'll wrangle, analyze, and visualize Statcast data to
compare Mr.~Judge and another (extremely large) teammate of his. Let's
start by loading the data into our Notebook. There are two CSV files,
judge.csv and stanton.csv, both of which contain Statcast data for
2015-2017. We'll use pandas DataFrames to store this data. Let's also
load data visualization libraries, matplotlib and seaborn.

    \begin{Verbatim}[commandchars=\\\{\}]
{\color{incolor}In [{\color{incolor}1}]:} \PY{c+c1}{\PYZsh{} import datavis libraries and pandas}
        \PY{k+kn}{import} \PY{n+nn}{pandas} \PY{k}{as} \PY{n+nn}{pd}
        \PY{k+kn}{import} \PY{n+nn}{matplotlib}\PY{n+nn}{.}\PY{n+nn}{pyplot} \PY{k}{as} \PY{n+nn}{plt}
        \PY{k+kn}{import} \PY{n+nn}{seaborn} \PY{k}{as} \PY{n+nn}{sns}
        \PY{o}{\PYZpc{}}\PY{k}{matplotlib} inline
        
        \PY{c+c1}{\PYZsh{} Load Aaron Judge\PYZsq{}s Statcast data}
        \PY{n}{judge} \PY{o}{=} \PY{n}{pd}\PY{o}{.}\PY{n}{read\PYZus{}csv}\PY{p}{(}\PY{l+s+s1}{\PYZsq{}}\PY{l+s+s1}{datasets/judge.csv}\PY{l+s+s1}{\PYZsq{}}\PY{p}{)}
        
        \PY{c+c1}{\PYZsh{} Load Giancarlo Stanton\PYZsq{}s Statcast data}
        \PY{n}{stanton} \PY{o}{=} \PY{n}{pd}\PY{o}{.}\PY{n}{read\PYZus{}csv}\PY{p}{(}\PY{l+s+s1}{\PYZsq{}}\PY{l+s+s1}{datasets/stanton.csv}\PY{l+s+s1}{\PYZsq{}}\PY{p}{)}
\end{Verbatim}


    \hypertarget{what-can-statcast-measure}{%
\subsection{2. What can Statcast
measure?}\label{what-can-statcast-measure}}

The better question might be, what can't Statcast measure?

Starting with the pitcher, Statcast can measure simple data points such
as velocity. At the same time, Statcast digs a whole lot deeper, also
measuring the release point and spin rate of every pitch.

Moving on to hitters, Statcast is capable of measuring the exit
velocity, launch angle and vector of the ball as it comes off the bat.
From there, Statcast can also track the hang time and projected distance
that a ball travels.

Let's inspect the last five rows of the judge DataFrame. You'll see that
each row represents one pitch thrown to a batter. You'll also see that
some columns have esoteric names. If these don't make sense now, don't
worry. The relevant ones will be explained as necessary.

    \begin{Verbatim}[commandchars=\\\{\}]
{\color{incolor}In [{\color{incolor}2}]:} \PY{c+c1}{\PYZsh{} Display all columns (pandas will collapse some columns if we don\PYZsq{}t set this option)}
        \PY{n}{pd}\PY{o}{.}\PY{n}{set\PYZus{}option}\PY{p}{(}\PY{l+s+s1}{\PYZsq{}}\PY{l+s+s1}{display.max\PYZus{}columns}\PY{l+s+s1}{\PYZsq{}}\PY{p}{,} \PY{k+kc}{None}\PY{p}{)}
        
        \PY{c+c1}{\PYZsh{} Display the last five rows of the Aaron Judge file}
        \PY{n+nb}{print}\PY{p}{(}\PY{n}{judge}\PY{o}{.}\PY{n}{tail}\PY{p}{(}\PY{p}{)}\PY{p}{)}
\end{Verbatim}


    \begin{Verbatim}[commandchars=\\\{\}]
     pitch\_type   game\_date  release\_speed  release\_pos\_x  release\_pos\_z  \textbackslash{}
3431         CH  2016-08-13           85.6        -1.9659         5.9113   
3432         CH  2016-08-13           87.6        -1.9318         5.9349   
3433         CH  2016-08-13           87.2        -2.0285         5.8656   
3434         CU  2016-08-13           79.7        -1.7108         6.1926   
3435         FF  2016-08-13           93.2        -1.8476         6.0063   

      player\_name  batter  pitcher    events          description  spin\_dir  \textbackslash{}
3431  Aaron Judge  592450   542882       NaN                 ball       NaN   
3432  Aaron Judge  592450   542882  home\_run  hit\_into\_play\_score       NaN   
3433  Aaron Judge  592450   542882       NaN                 ball       NaN   
3434  Aaron Judge  592450   542882       NaN                 foul       NaN   
3435  Aaron Judge  592450   542882       NaN        called\_strike       NaN   

      spin\_rate\_deprecated  break\_angle\_deprecated  break\_length\_deprecated  \textbackslash{}
3431                   NaN                     NaN                      NaN   
3432                   NaN                     NaN                      NaN   
3433                   NaN                     NaN                      NaN   
3434                   NaN                     NaN                      NaN   
3435                   NaN                     NaN                      NaN   

      zone                                                des game\_type stand  \textbackslash{}
3431  14.0                                                NaN         R     R   
3432   4.0  Aaron Judge homers (1) on a fly ball to center{\ldots}         R     R   
3433  14.0                                                NaN         R     R   
3434   4.0                                                NaN         R     R   
3435   8.0                                                NaN         R     R   

     p\_throws home\_team away\_team type  hit\_location   bb\_type  balls  \textbackslash{}
3431        R       NYY        TB    B           NaN       NaN      0   
3432        R       NYY        TB    X           NaN  fly\_ball      1   
3433        R       NYY        TB    B           NaN       NaN      0   
3434        R       NYY        TB    S           NaN       NaN      0   
3435        R       NYY        TB    S           NaN       NaN      0   

      strikes  game\_year     pfx\_x     pfx\_z  plate\_x  plate\_z  on\_3b  on\_2b  \textbackslash{}
3431        0       2016 -0.379108  0.370567    0.739    1.442    NaN    NaN   
3432        2       2016 -0.295608  0.320400   -0.419    3.273    NaN    NaN   
3433        2       2016 -0.668575  0.198567    0.561    0.960    NaN    NaN   
3434        1       2016  0.397442 -0.614133   -0.803    2.742    NaN    NaN   
3435        0       2016 -0.823050  1.623300   -0.273    2.471    NaN    NaN   

      on\_1b  outs\_when\_up  inning inning\_topbot    hc\_x   hc\_y  \textbackslash{}
3431    NaN             0       5           Bot     NaN    NaN   
3432    NaN             2       2           Bot  130.45  14.58   
3433    NaN             2       2           Bot     NaN    NaN   
3434    NaN             2       2           Bot     NaN    NaN   
3435    NaN             2       2           Bot     NaN    NaN   

      tfs\_deprecated  tfs\_zulu\_deprecated  pos2\_person\_id  umpire  \textbackslash{}
3431             NaN                  NaN        571912.0     NaN   
3432             NaN                  NaN        571912.0     NaN   
3433             NaN                  NaN        571912.0     NaN   
3434             NaN                  NaN        571912.0     NaN   
3435             NaN                  NaN        571912.0     NaN   

              sv\_id    vx0      vy0    vz0     ax      ay      az  sz\_top  \textbackslash{}
3431  160813\_144259  6.960 -124.371 -4.756 -2.821  23.634 -30.220    3.93   
3432  160813\_135833  4.287 -127.452 -0.882 -1.972  24.694 -30.705    4.01   
3433  160813\_135815  7.491 -126.665 -5.862 -6.393  21.952 -32.121    4.01   
3434  160813\_135752  1.254 -116.062  0.439  5.184  21.328 -39.866    4.01   
3435  160813\_135736  5.994 -135.497 -6.736 -9.360  26.782 -13.446    4.01   

      sz\_bot  hit\_distance\_sc  launch\_speed  launch\_angle  effective\_speed  \textbackslash{}
3431    1.82              NaN           NaN           NaN           84.459   
3432    1.82            446.0         108.8        27.410           86.412   
3433    1.82              NaN           NaN           NaN           86.368   
3434    1.82              9.0          55.8       -24.973           77.723   
3435    1.82              NaN           NaN           NaN           92.696   

      release\_spin\_rate  release\_extension  game\_pk  pos1\_person\_id  \textbackslash{}
3431             1552.0              5.683   448611        542882.0   
3432             1947.0              5.691   448611        542882.0   
3433             1761.0              5.721   448611        542882.0   
3434             2640.0              5.022   448611        542882.0   
3435             2271.0              6.068   448611        542882.0   

      pos2\_person\_id.1  pos3\_person\_id  pos4\_person\_id  pos5\_person\_id  \textbackslash{}
3431          571912.0        543543.0        523253.0        446334.0   
3432          571912.0        543543.0        523253.0        446334.0   
3433          571912.0        543543.0        523253.0        446334.0   
3434          571912.0        543543.0        523253.0        446334.0   
3435          571912.0        543543.0        523253.0        446334.0   

      pos6\_person\_id  pos7\_person\_id  pos8\_person\_id  pos9\_person\_id  \textbackslash{}
3431        622110.0        545338.0        595281.0        543484.0   
3432        622110.0        545338.0        595281.0        543484.0   
3433        622110.0        545338.0        595281.0        543484.0   
3434        622110.0        545338.0        595281.0        543484.0   
3435        622110.0        545338.0        595281.0        543484.0   

      release\_pos\_y  estimated\_ba\_using\_speedangle  \textbackslash{}
3431        54.8144                           0.00   
3432        54.8064                           0.98   
3433        54.7770                           0.00   
3434        55.4756                           0.00   
3435        54.4299                           0.00   

      estimated\_woba\_using\_speedangle  woba\_value  woba\_denom  babip\_value  \textbackslash{}
3431                            0.000         NaN         NaN          NaN   
3432                            1.937         2.0         1.0          0.0   
3433                            0.000         NaN         NaN          NaN   
3434                            0.000         NaN         NaN          NaN   
3435                            0.000         NaN         NaN          NaN   

      iso\_value  launch\_speed\_angle  at\_bat\_number  pitch\_number  
3431        NaN                 NaN             36             1  
3432        3.0                 6.0             14             4  
3433        NaN                 NaN             14             3  
3434        NaN                 1.0             14             2  
3435        NaN                 NaN             14             1  

    \end{Verbatim}

    \hypertarget{aaron-judge-and-giancarlo-stanton-prolific-sluggers}{%
\subsection{3. Aaron Judge and Giancarlo Stanton, prolific
sluggers}\label{aaron-judge-and-giancarlo-stanton-prolific-sluggers}}

This is Giancarlo Stanton. He is also a very large human being, standing
6 feet 6 inches tall and weighing 245 pounds. Despite not wearing the
same jersey as Judge in the pictures provided, in 2018 they will be
teammates on the New York Yankees. They are similar in a lot of ways,
one being that they hit a lot of home runs. Stanton and Judge led
baseball in home runs in 2017, with 59 and 52, respectively. These are
exceptional totals - the player in third ``only'' had 45 home runs.

Stanton and Judge are also different in many ways. One is batted ball
events, which is any batted ball that produces a result. This includes
outs, hits, and errors. Next, you'll find the counts of batted ball
events for each player in 2017. The frequencies of other events are
quite different.

    \begin{Verbatim}[commandchars=\\\{\}]
{\color{incolor}In [{\color{incolor}3}]:} \PY{c+c1}{\PYZsh{} All of Aaron Judge\PYZsq{}s batted ball events in 2017}
        \PY{n}{judge\PYZus{}events\PYZus{}2017} \PY{o}{=} \PY{n}{judge}\PY{o}{.}\PY{n}{loc}\PY{p}{[}\PY{n}{judge}\PY{p}{[}\PY{l+s+s1}{\PYZsq{}}\PY{l+s+s1}{game\PYZus{}year}\PY{l+s+s1}{\PYZsq{}}\PY{p}{]} \PY{o}{==} \PY{l+m+mi}{2017}\PY{p}{]}\PY{o}{.}\PY{n}{events}
        \PY{n+nb}{print}\PY{p}{(}\PY{l+s+s2}{\PYZdq{}}\PY{l+s+s2}{Aaron Judge batted ball event totals, 2017: }\PY{l+s+s2}{\PYZdq{}}\PY{p}{)}
        \PY{n+nb}{print}\PY{p}{(}\PY{n}{judge\PYZus{}events\PYZus{}2017}\PY{o}{.}\PY{n}{value\PYZus{}counts}\PY{p}{(}\PY{p}{)}\PY{p}{)}
        
        \PY{c+c1}{\PYZsh{} All of Giancarlo Stanton\PYZsq{}s batted ball events in 2017}
        \PY{n}{stanton\PYZus{}events\PYZus{}2017} \PY{o}{=} \PY{n}{stanton}\PY{o}{.}\PY{n}{loc}\PY{p}{[}\PY{n}{stanton}\PY{p}{[}\PY{l+s+s1}{\PYZsq{}}\PY{l+s+s1}{game\PYZus{}year}\PY{l+s+s1}{\PYZsq{}}\PY{p}{]} \PY{o}{==} \PY{l+m+mi}{2017}\PY{p}{]}\PY{o}{.}\PY{n}{events}
        \PY{n+nb}{print}\PY{p}{(}\PY{l+s+s2}{\PYZdq{}}\PY{l+s+se}{\PYZbs{}n}\PY{l+s+s2}{Giancarlo Stanton batted ball event totals, 2017: }\PY{l+s+s2}{\PYZdq{}}\PY{p}{)}
        \PY{n+nb}{print}\PY{p}{(}\PY{n}{stanton\PYZus{}events\PYZus{}2017}\PY{o}{.}\PY{n}{value\PYZus{}counts}\PY{p}{(}\PY{p}{)}\PY{p}{)}
\end{Verbatim}


    \begin{Verbatim}[commandchars=\\\{\}]
Aaron Judge batted ball event totals, 2017: 
strikeout                    207
field\_out                    146
walk                         116
single                        75
home\_run                      52
double                        24
grounded\_into\_double\_play     15
force\_out                     11
intent\_walk                   11
hit\_by\_pitch                   5
fielders\_choice\_out            4
field\_error                    4
sac\_fly                        4
triple                         3
strikeout\_double\_play          1
Name: events, dtype: int64

Giancarlo Stanton batted ball event totals, 2017: 
field\_out                    239
strikeout                    161
single                        77
walk                          72
home\_run                      59
double                        32
grounded\_into\_double\_play     13
intent\_walk                   13
hit\_by\_pitch                   7
force\_out                      7
field\_error                    5
sac\_fly                        3
fielders\_choice\_out            2
strikeout\_double\_play          2
pickoff\_1b                     1
Name: events, dtype: int64

    \end{Verbatim}

    \hypertarget{analyzing-home-runs-with-statcast-data}{%
\subsection{4. Analyzing home runs with Statcast
data}\label{analyzing-home-runs-with-statcast-data}}

So Judge walks and strikes out more than Stanton. Stanton flies out more
than Judge. But let's get into their hitting profiles in more detail.
Two of the most groundbreaking Statcast metrics are launch angle and
exit velocity:

Launch angle: the vertical angle at which the ball leaves a player's bat

Exit velocity: the speed of the baseball as it comes off the bat

This new data has changed the way teams value both hitters and pitchers.
Why? As per the Washington Post:

Balls hit with a high launch angle are more likely to result in a hit.
Hit fast enough and at the right angle, they become home runs.

Let's look at exit velocity vs.~launch angle and let's focus on home
runs only (2015-2017). The first two plots show data points. The second
two show smoothed contours to represent density.

    \begin{Verbatim}[commandchars=\\\{\}]
{\color{incolor}In [{\color{incolor}4}]:} \PY{c+c1}{\PYZsh{} Filter to include home runs only}
        \PY{n}{judge\PYZus{}hr} \PY{o}{=} \PY{n}{judge}\PY{o}{.}\PY{n}{loc}\PY{p}{[}\PY{p}{(}\PY{n}{judge}\PY{p}{[}\PY{l+s+s1}{\PYZsq{}}\PY{l+s+s1}{game\PYZus{}year}\PY{l+s+s1}{\PYZsq{}}\PY{p}{]} \PY{o}{\PYZgt{}}\PY{o}{=} \PY{l+m+mi}{2015}\PY{p}{)} \PY{o}{\PYZam{}} \PY{p}{(}\PY{n}{judge}\PY{p}{[}\PY{l+s+s1}{\PYZsq{}}\PY{l+s+s1}{game\PYZus{}year}\PY{l+s+s1}{\PYZsq{}}\PY{p}{]} \PY{o}{\PYZlt{}}\PY{o}{=} \PY{l+m+mi}{2017}\PY{p}{)} \PY{o}{\PYZam{}} \PY{p}{(}\PY{n}{judge}\PY{p}{[}\PY{l+s+s1}{\PYZsq{}}\PY{l+s+s1}{events}\PY{l+s+s1}{\PYZsq{}}\PY{p}{]} \PY{o}{==} \PY{l+s+s1}{\PYZsq{}}\PY{l+s+s1}{home\PYZus{}run}\PY{l+s+s1}{\PYZsq{}}\PY{p}{)}\PY{p}{]}
        \PY{n}{stanton\PYZus{}hr} \PY{o}{=} \PY{n}{stanton}\PY{o}{.}\PY{n}{loc}\PY{p}{[}\PY{p}{(}\PY{n}{stanton}\PY{p}{[}\PY{l+s+s1}{\PYZsq{}}\PY{l+s+s1}{game\PYZus{}year}\PY{l+s+s1}{\PYZsq{}}\PY{p}{]} \PY{o}{\PYZgt{}}\PY{o}{=} \PY{l+m+mi}{2015}\PY{p}{)} \PY{o}{\PYZam{}} \PY{p}{(}\PY{n}{stanton}\PY{p}{[}\PY{l+s+s1}{\PYZsq{}}\PY{l+s+s1}{game\PYZus{}year}\PY{l+s+s1}{\PYZsq{}}\PY{p}{]} \PY{o}{\PYZlt{}}\PY{o}{=} \PY{l+m+mi}{2017}\PY{p}{)} \PY{o}{\PYZam{}} \PY{p}{(}\PY{n}{stanton}\PY{p}{[}\PY{l+s+s1}{\PYZsq{}}\PY{l+s+s1}{events}\PY{l+s+s1}{\PYZsq{}}\PY{p}{]} \PY{o}{==} \PY{l+s+s1}{\PYZsq{}}\PY{l+s+s1}{home\PYZus{}run}\PY{l+s+s1}{\PYZsq{}}\PY{p}{)}\PY{p}{]}
        
        \PY{c+c1}{\PYZsh{} Create a figure with two scatter plots of launch speed vs. launch angle, one for each player\PYZsq{}s home runs}
        \PY{n}{fig1}\PY{p}{,} \PY{n}{axs1} \PY{o}{=} \PY{n}{plt}\PY{o}{.}\PY{n}{subplots}\PY{p}{(}\PY{n}{ncols}\PY{o}{=}\PY{l+m+mi}{2}\PY{p}{,} \PY{n}{sharex}\PY{o}{=}\PY{k+kc}{True}\PY{p}{,} \PY{n}{sharey}\PY{o}{=}\PY{k+kc}{True}\PY{p}{)}
        \PY{n}{sns}\PY{o}{.}\PY{n}{regplot}\PY{p}{(}\PY{n}{x}\PY{o}{=}\PY{n}{judge\PYZus{}hr}\PY{p}{[}\PY{l+s+s1}{\PYZsq{}}\PY{l+s+s1}{launch\PYZus{}speed}\PY{l+s+s1}{\PYZsq{}}\PY{p}{]}\PY{p}{,} \PY{n}{y}\PY{o}{=}\PY{n}{judge\PYZus{}hr}\PY{p}{[}\PY{l+s+s1}{\PYZsq{}}\PY{l+s+s1}{launch\PYZus{}angle}\PY{l+s+s1}{\PYZsq{}}\PY{p}{]}\PY{p}{,} \PY{n}{fit\PYZus{}reg}\PY{o}{=}\PY{k+kc}{False}\PY{p}{,} \PY{n}{color}\PY{o}{=}\PY{l+s+s1}{\PYZsq{}}\PY{l+s+s1}{tab:blue}\PY{l+s+s1}{\PYZsq{}}\PY{p}{,} \PY{n}{data}\PY{o}{=}\PY{n}{judge\PYZus{}hr}\PY{p}{,} \PY{n}{ax}\PY{o}{=}\PY{n}{axs1}\PY{p}{[}\PY{l+m+mi}{0}\PY{p}{]}\PY{p}{)}\PY{o}{.}\PY{n}{set\PYZus{}title}\PY{p}{(}\PY{l+s+s1}{\PYZsq{}}\PY{l+s+s1}{Aaron Judge}\PY{l+s+se}{\PYZbs{}n}\PY{l+s+s1}{Home Runs, 2015\PYZhy{}2017}\PY{l+s+s1}{\PYZsq{}}\PY{p}{)}
        \PY{n}{sns}\PY{o}{.}\PY{n}{regplot}\PY{p}{(}\PY{n}{x}\PY{o}{=}\PY{n}{stanton\PYZus{}hr}\PY{p}{[}\PY{l+s+s1}{\PYZsq{}}\PY{l+s+s1}{launch\PYZus{}speed}\PY{l+s+s1}{\PYZsq{}}\PY{p}{]}\PY{p}{,} \PY{n}{y}\PY{o}{=}\PY{n}{stanton\PYZus{}hr}\PY{p}{[}\PY{l+s+s1}{\PYZsq{}}\PY{l+s+s1}{launch\PYZus{}angle}\PY{l+s+s1}{\PYZsq{}}\PY{p}{]}\PY{p}{,} \PY{n}{fit\PYZus{}reg}\PY{o}{=}\PY{k+kc}{False}\PY{p}{,} \PY{n}{color}\PY{o}{=}\PY{l+s+s1}{\PYZsq{}}\PY{l+s+s1}{tab:blue}\PY{l+s+s1}{\PYZsq{}}\PY{p}{,} \PY{n}{data}\PY{o}{=}\PY{n}{stanton\PYZus{}hr}\PY{p}{,} \PY{n}{ax}\PY{o}{=}\PY{n}{axs1}\PY{p}{[}\PY{l+m+mi}{1}\PY{p}{]}\PY{p}{)}\PY{o}{.}\PY{n}{set\PYZus{}title}\PY{p}{(}\PY{l+s+s1}{\PYZsq{}}\PY{l+s+s1}{Giancarlo Stanton}\PY{l+s+se}{\PYZbs{}n}\PY{l+s+s1}{Home Runs, 2015\PYZhy{}2017}\PY{l+s+s1}{\PYZsq{}}\PY{p}{)}
        
        \PY{c+c1}{\PYZsh{} Create a figure with two KDE plots of launch speed vs. launch angle, one for each player\PYZsq{}s home runs}
        \PY{n}{fig2}\PY{p}{,} \PY{n}{axs2} \PY{o}{=} \PY{n}{plt}\PY{o}{.}\PY{n}{subplots}\PY{p}{(}\PY{n}{ncols}\PY{o}{=}\PY{l+m+mi}{2}\PY{p}{,} \PY{n}{sharex}\PY{o}{=}\PY{k+kc}{True}\PY{p}{,} \PY{n}{sharey}\PY{o}{=}\PY{k+kc}{True}\PY{p}{)}
        \PY{n}{sns}\PY{o}{.}\PY{n}{kdeplot}\PY{p}{(}\PY{n}{judge\PYZus{}hr}\PY{p}{[}\PY{l+s+s1}{\PYZsq{}}\PY{l+s+s1}{launch\PYZus{}speed}\PY{l+s+s1}{\PYZsq{}}\PY{p}{]}\PY{p}{,} \PY{n}{judge\PYZus{}hr}\PY{p}{[}\PY{l+s+s1}{\PYZsq{}}\PY{l+s+s1}{launch\PYZus{}angle}\PY{l+s+s1}{\PYZsq{}}\PY{p}{]}\PY{p}{,} \PY{n}{cmap}\PY{o}{=}\PY{l+s+s2}{\PYZdq{}}\PY{l+s+s2}{Blues}\PY{l+s+s2}{\PYZdq{}}\PY{p}{,} \PY{n}{shade}\PY{o}{=}\PY{k+kc}{True}\PY{p}{,} \PY{n}{shade\PYZus{}lowest}\PY{o}{=}\PY{k+kc}{False}\PY{p}{,} \PY{n}{ax}\PY{o}{=}\PY{n}{axs2}\PY{p}{[}\PY{l+m+mi}{0}\PY{p}{]}\PY{p}{)}\PY{o}{.}\PY{n}{set\PYZus{}title}\PY{p}{(}\PY{l+s+s1}{\PYZsq{}}\PY{l+s+s1}{Aaron Judge}\PY{l+s+se}{\PYZbs{}n}\PY{l+s+s1}{Home Runs, 2015\PYZhy{}2017}\PY{l+s+s1}{\PYZsq{}}\PY{p}{)}
        \PY{n}{sns}\PY{o}{.}\PY{n}{kdeplot}\PY{p}{(}\PY{n}{stanton\PYZus{}hr}\PY{p}{[}\PY{l+s+s1}{\PYZsq{}}\PY{l+s+s1}{launch\PYZus{}speed}\PY{l+s+s1}{\PYZsq{}}\PY{p}{]}\PY{p}{,} \PY{n}{stanton\PYZus{}hr}\PY{p}{[}\PY{l+s+s1}{\PYZsq{}}\PY{l+s+s1}{launch\PYZus{}angle}\PY{l+s+s1}{\PYZsq{}}\PY{p}{]}\PY{p}{,} \PY{n}{cmap}\PY{o}{=}\PY{l+s+s2}{\PYZdq{}}\PY{l+s+s2}{Blues}\PY{l+s+s2}{\PYZdq{}}\PY{p}{,} \PY{n}{shade}\PY{o}{=}\PY{k+kc}{True}\PY{p}{,} \PY{n}{shade\PYZus{}lowest}\PY{o}{=}\PY{k+kc}{False}\PY{p}{,} \PY{n}{ax}\PY{o}{=}\PY{n}{axs2}\PY{p}{[}\PY{l+m+mi}{1}\PY{p}{]}\PY{p}{)}\PY{o}{.}\PY{n}{set\PYZus{}title}\PY{p}{(}\PY{l+s+s1}{\PYZsq{}}\PY{l+s+s1}{Giancarlo Stanton}\PY{l+s+se}{\PYZbs{}n}\PY{l+s+s1}{Home Runs, 2015\PYZhy{}2017}\PY{l+s+s1}{\PYZsq{}}\PY{p}{)}
\end{Verbatim}


    \begin{Verbatim}[commandchars=\\\{\}]
/home/smcdnyc/anaconda3/lib/python3.7/site-packages/scipy/stats/stats.py:1713: FutureWarning: Using a non-tuple sequence for multidimensional indexing is deprecated; use `arr[tuple(seq)]` instead of `arr[seq]`. In the future this will be interpreted as an array index, `arr[np.array(seq)]`, which will result either in an error or a different result.
  return np.add.reduce(sorted[indexer] * weights, axis=axis) / sumval

    \end{Verbatim}

\begin{Verbatim}[commandchars=\\\{\}]
{\color{outcolor}Out[{\color{outcolor}4}]:} Text(0.5,1,'Giancarlo Stanton\textbackslash{}nHome Runs, 2015-2017')
\end{Verbatim}
            
    \begin{center}
    \adjustimage{max size={0.9\linewidth}{0.9\paperheight}}{output_7_2.png}
    \end{center}
    { \hspace*{\fill} \\}
    
    \begin{center}
    \adjustimage{max size={0.9\linewidth}{0.9\paperheight}}{output_7_3.png}
    \end{center}
    { \hspace*{\fill} \\}
    
    \hypertarget{home-runs-by-pitch-velocity}{%
\subsection{5. Home runs by pitch
velocity}\label{home-runs-by-pitch-velocity}}

It appears that Stanton hits his home runs slightly lower and slightly
harder than Judge, though this needs to be taken with a grain of salt
given the small sample size of home runs.

Not only does Statcast measure the velocity of the ball coming off of
the bat, it measures the velocity of the ball coming out of the
pitcher's hand and begins its journey towards the plate. We can use this
data to compare Stanton and Judge's home runs in terms of pitch
velocity. Next you'll find box plots displaying the five-number
summaries for each player: minimum, first quartile, median, third
quartile, and maximum.

    \begin{Verbatim}[commandchars=\\\{\}]
{\color{incolor}In [{\color{incolor}5}]:} \PY{c+c1}{\PYZsh{} Combine the Judge and Stanton home run DataFrames for easy boxplot plotting}
        \PY{n}{judge\PYZus{}stanton\PYZus{}hr} \PY{o}{=} \PY{n}{pd}\PY{o}{.}\PY{n}{concat}\PY{p}{(}\PY{p}{[}\PY{n}{judge\PYZus{}hr}\PY{p}{,} \PY{n}{stanton\PYZus{}hr}\PY{p}{]}\PY{p}{)}
        
        \PY{c+c1}{\PYZsh{} Create a boxplot that describes the pitch velocity of each player\PYZsq{}s home runs}
        \PY{n}{sns}\PY{o}{.}\PY{n}{boxplot}\PY{p}{(}\PY{n}{x}\PY{o}{=}\PY{n}{judge\PYZus{}stanton\PYZus{}hr}\PY{p}{[}\PY{l+s+s1}{\PYZsq{}}\PY{l+s+s1}{player\PYZus{}name}\PY{l+s+s1}{\PYZsq{}}\PY{p}{]}\PY{p}{,} \PY{n}{y}\PY{o}{=}\PY{n}{judge\PYZus{}stanton\PYZus{}hr}\PY{p}{[}\PY{l+s+s1}{\PYZsq{}}\PY{l+s+s1}{release\PYZus{}speed}\PY{l+s+s1}{\PYZsq{}}\PY{p}{]}\PY{p}{,} \PY{n}{color}\PY{o}{=}\PY{l+s+s1}{\PYZsq{}}\PY{l+s+s1}{tab:blue}\PY{l+s+s1}{\PYZsq{}}\PY{p}{)}\PY{o}{.}\PY{n}{set\PYZus{}title}\PY{p}{(}\PY{l+s+s1}{\PYZsq{}}\PY{l+s+s1}{Home Runs, 2015\PYZhy{}2017}\PY{l+s+s1}{\PYZsq{}}\PY{p}{)}
\end{Verbatim}


\begin{Verbatim}[commandchars=\\\{\}]
{\color{outcolor}Out[{\color{outcolor}5}]:} Text(0.5,1,'Home Runs, 2015-2017')
\end{Verbatim}
            
    \begin{center}
    \adjustimage{max size={0.9\linewidth}{0.9\paperheight}}{output_9_1.png}
    \end{center}
    { \hspace*{\fill} \\}
    
    \hypertarget{home-runs-by-pitch-location-i}{%
\subsection{6. Home runs by pitch location
(I)}\label{home-runs-by-pitch-location-i}}

So Judge appears to hit his home runs off of faster pitches than
Stanton. We might call Judge a fastball hitter. Stanton appears agnostic
to pitch speed and likely pitch movement since slower pitches
(e.g.~curveballs, sliders, and changeups) tend to have more break.
Statcast does track pitch movement and type but let's move on to
something else: pitch location. Statcast tracks the zone the pitch is in
when it crosses the plate. The zone numbering looks like this (from the
catcher's point of view):

We can plot this using a 2D histogram. For simplicity, let's only look
at strikes, which gives us a 9x9 grid. We can view each zone as
coordinates on a 2D plot, the bottom left corner being (1,1) and the top
right corner being (3,3). Let's set up a function to assign
x-coordinates to each pitch.

    \begin{Verbatim}[commandchars=\\\{\}]
{\color{incolor}In [{\color{incolor}6}]:} \PY{k}{def} \PY{n+nf}{assign\PYZus{}x\PYZus{}coord}\PY{p}{(}\PY{n}{row}\PY{p}{)}\PY{p}{:}
            \PY{l+s+sd}{\PYZdq{}\PYZdq{}\PYZdq{}}
        \PY{l+s+sd}{    Assigns an x\PYZhy{}coordinate to Statcast\PYZsq{}s strike zone numbers. Zones 11, 12, 13,}
        \PY{l+s+sd}{    and 14 are ignored for plotting simplicity.}
        \PY{l+s+sd}{    \PYZdq{}\PYZdq{}\PYZdq{}}
            \PY{c+c1}{\PYZsh{} Left third of strike zone}
            \PY{k}{if} \PY{n}{row}\PY{o}{.}\PY{n}{zone} \PY{o+ow}{in} \PY{p}{[}\PY{l+m+mi}{1}\PY{p}{,} \PY{l+m+mi}{4}\PY{p}{,} \PY{l+m+mi}{7}\PY{p}{]}\PY{p}{:}
                \PY{k}{return} \PY{l+m+mi}{1}
            \PY{c+c1}{\PYZsh{} Middle third of strike zone}
            \PY{k}{if} \PY{n}{row}\PY{o}{.}\PY{n}{zone} \PY{o+ow}{in} \PY{p}{[}\PY{l+m+mi}{2}\PY{p}{,} \PY{l+m+mi}{5}\PY{p}{,} \PY{l+m+mi}{8}\PY{p}{]}\PY{p}{:}
                \PY{k}{return} \PY{l+m+mi}{2}
            \PY{c+c1}{\PYZsh{} Right third of strike zone}
            \PY{k}{if} \PY{n}{row}\PY{o}{.}\PY{n}{zone} \PY{o+ow}{in} \PY{p}{[}\PY{l+m+mi}{3}\PY{p}{,} \PY{l+m+mi}{6}\PY{p}{,} \PY{l+m+mi}{9}\PY{p}{]}\PY{p}{:}
                \PY{k}{return} \PY{l+m+mi}{3}
\end{Verbatim}


    \hypertarget{home-runs-by-pitch-location-ii}{%
\subsection{7. Home runs by pitch location
(II)}\label{home-runs-by-pitch-location-ii}}

And let's do the same but for y-coordinates.

    \begin{Verbatim}[commandchars=\\\{\}]
{\color{incolor}In [{\color{incolor}7}]:} \PY{k}{def} \PY{n+nf}{assign\PYZus{}y\PYZus{}coord}\PY{p}{(}\PY{n}{row}\PY{p}{)}\PY{p}{:}
            \PY{l+s+sd}{\PYZdq{}\PYZdq{}\PYZdq{}}
        \PY{l+s+sd}{    Assigns a y\PYZhy{}coordinate to Statcast\PYZsq{}s strike zone numbers. Zones 11, 12, 13,}
        \PY{l+s+sd}{    and 14 are ignored for plotting simplicity.}
        \PY{l+s+sd}{    \PYZdq{}\PYZdq{}\PYZdq{}}
            \PY{c+c1}{\PYZsh{} Upper third of strike zone}
            \PY{k}{if} \PY{n}{row}\PY{o}{.}\PY{n}{zone} \PY{o+ow}{in} \PY{p}{[}\PY{l+m+mi}{1}\PY{p}{,} \PY{l+m+mi}{2}\PY{p}{,} \PY{l+m+mi}{3}\PY{p}{]}\PY{p}{:}
                \PY{k}{return} \PY{l+m+mi}{3}
            \PY{c+c1}{\PYZsh{} Middle third of strike zone}
            \PY{k}{if} \PY{n}{row}\PY{o}{.}\PY{n}{zone} \PY{o+ow}{in} \PY{p}{[}\PY{l+m+mi}{4}\PY{p}{,} \PY{l+m+mi}{5}\PY{p}{,} \PY{l+m+mi}{6}\PY{p}{]}\PY{p}{:}
                \PY{k}{return} \PY{l+m+mi}{2}
            \PY{c+c1}{\PYZsh{} Lower third of strike zone}
            \PY{k}{if} \PY{n}{row}\PY{o}{.}\PY{n}{zone} \PY{o+ow}{in} \PY{p}{[}\PY{l+m+mi}{7}\PY{p}{,} \PY{l+m+mi}{8}\PY{p}{,} \PY{l+m+mi}{9}\PY{p}{]}\PY{p}{:}
                \PY{k}{return} \PY{l+m+mi}{1}
\end{Verbatim}


    \hypertarget{aaron-judges-home-run-zone}{%
\subsection{8. Aaron Judge's home run
zone}\label{aaron-judges-home-run-zone}}

Now we can apply the functions we've created then construct our 2D
histograms. First, for Aaron Judge (again, for pitches in the strike
zone that resulted in home runs).

    \begin{Verbatim}[commandchars=\\\{\}]
{\color{incolor}In [{\color{incolor}8}]:} \PY{c+c1}{\PYZsh{} Zones 11, 12, 13, and 14 are to be ignored for plotting simplicity}
        \PY{n}{judge\PYZus{}strike\PYZus{}hr} \PY{o}{=} \PY{n}{judge\PYZus{}hr}\PY{o}{.}\PY{n}{copy}\PY{p}{(}\PY{p}{)}\PY{o}{.}\PY{n}{loc}\PY{p}{[}\PY{n}{judge\PYZus{}hr}\PY{o}{.}\PY{n}{zone} \PY{o}{\PYZlt{}}\PY{o}{=} \PY{l+m+mi}{9}\PY{p}{]}
        
        \PY{c+c1}{\PYZsh{} Assign Cartesian coordinates to pitches in the strike zone for Judge home runs}
        \PY{n}{judge\PYZus{}strike\PYZus{}hr}\PY{p}{[}\PY{l+s+s1}{\PYZsq{}}\PY{l+s+s1}{zone\PYZus{}x}\PY{l+s+s1}{\PYZsq{}}\PY{p}{]} \PY{o}{=} \PY{n}{judge\PYZus{}strike\PYZus{}hr}\PY{o}{.}\PY{n}{apply}\PY{p}{(}\PY{n}{assign\PYZus{}x\PYZus{}coord}\PY{p}{,} \PY{n}{axis} \PY{o}{=} \PY{l+m+mi}{1}\PY{p}{)}
        \PY{n}{judge\PYZus{}strike\PYZus{}hr}\PY{p}{[}\PY{l+s+s1}{\PYZsq{}}\PY{l+s+s1}{zone\PYZus{}y}\PY{l+s+s1}{\PYZsq{}}\PY{p}{]} \PY{o}{=} \PY{n}{judge\PYZus{}strike\PYZus{}hr}\PY{o}{.}\PY{n}{apply}\PY{p}{(}\PY{n}{assign\PYZus{}y\PYZus{}coord}\PY{p}{,} \PY{n}{axis} \PY{o}{=} \PY{l+m+mi}{1}\PY{p}{)}
        
        \PY{c+c1}{\PYZsh{} Plot Judge\PYZsq{}s home run zone as a 2D histogram with a colorbar}
        \PY{n}{plt}\PY{o}{.}\PY{n}{hist2d}\PY{p}{(}\PY{n}{judge\PYZus{}strike\PYZus{}hr}\PY{p}{[}\PY{l+s+s1}{\PYZsq{}}\PY{l+s+s1}{zone\PYZus{}x}\PY{l+s+s1}{\PYZsq{}}\PY{p}{]}\PY{p}{,} \PY{n}{judge\PYZus{}strike\PYZus{}hr}\PY{p}{[}\PY{l+s+s1}{\PYZsq{}}\PY{l+s+s1}{zone\PYZus{}y}\PY{l+s+s1}{\PYZsq{}}\PY{p}{]}\PY{p}{,} \PY{n}{bins} \PY{o}{=} \PY{l+m+mi}{3}\PY{p}{,} \PY{n}{cmap}\PY{o}{=}\PY{l+s+s1}{\PYZsq{}}\PY{l+s+s1}{Blues}\PY{l+s+s1}{\PYZsq{}}\PY{p}{)}
        \PY{n}{plt}\PY{o}{.}\PY{n}{title}\PY{p}{(}\PY{l+s+s1}{\PYZsq{}}\PY{l+s+s1}{Aaron Judge Home Runs on}\PY{l+s+se}{\PYZbs{}n}\PY{l+s+s1}{ Pitches in the Strike Zone, 2015\PYZhy{}2017}\PY{l+s+s1}{\PYZsq{}}\PY{p}{)}
        \PY{n}{plt}\PY{o}{.}\PY{n}{gca}\PY{p}{(}\PY{p}{)}\PY{o}{.}\PY{n}{get\PYZus{}xaxis}\PY{p}{(}\PY{p}{)}\PY{o}{.}\PY{n}{set\PYZus{}visible}\PY{p}{(}\PY{k+kc}{False}\PY{p}{)}
        \PY{n}{plt}\PY{o}{.}\PY{n}{gca}\PY{p}{(}\PY{p}{)}\PY{o}{.}\PY{n}{get\PYZus{}yaxis}\PY{p}{(}\PY{p}{)}\PY{o}{.}\PY{n}{set\PYZus{}visible}\PY{p}{(}\PY{k+kc}{False}\PY{p}{)}
        \PY{n}{cb} \PY{o}{=} \PY{n}{plt}\PY{o}{.}\PY{n}{colorbar}\PY{p}{(}\PY{p}{)}
        \PY{n}{cb}\PY{o}{.}\PY{n}{set\PYZus{}label}\PY{p}{(}\PY{l+s+s1}{\PYZsq{}}\PY{l+s+s1}{Counts in Bin}\PY{l+s+s1}{\PYZsq{}}\PY{p}{)}
\end{Verbatim}


    \begin{center}
    \adjustimage{max size={0.9\linewidth}{0.9\paperheight}}{output_15_0.png}
    \end{center}
    { \hspace*{\fill} \\}
    
    \hypertarget{giancarlo-stantons-home-run-zone}{%
\subsection{9. Giancarlo Stanton's home run
zone}\label{giancarlo-stantons-home-run-zone}}

And now for Giancarlo Stanton.

    \begin{Verbatim}[commandchars=\\\{\}]
{\color{incolor}In [{\color{incolor}9}]:} \PY{c+c1}{\PYZsh{} Zones 11, 12, 13, and 14 are to be ignored for plotting simplicity}
        \PY{n}{stanton\PYZus{}strike\PYZus{}hr} \PY{o}{=} \PY{n}{stanton\PYZus{}hr}\PY{o}{.}\PY{n}{copy}\PY{p}{(}\PY{p}{)}\PY{o}{.}\PY{n}{loc}\PY{p}{[}\PY{n}{stanton\PYZus{}hr}\PY{o}{.}\PY{n}{zone} \PY{o}{\PYZlt{}}\PY{o}{=} \PY{l+m+mi}{9}\PY{p}{]}
        
        \PY{c+c1}{\PYZsh{} Assign Cartesian coordinates to pitches in the strike zone for Stanton home runs}
        \PY{n}{stanton\PYZus{}strike\PYZus{}hr}\PY{p}{[}\PY{l+s+s1}{\PYZsq{}}\PY{l+s+s1}{zone\PYZus{}x}\PY{l+s+s1}{\PYZsq{}}\PY{p}{]} \PY{o}{=} \PY{n}{stanton\PYZus{}strike\PYZus{}hr}\PY{o}{.}\PY{n}{apply}\PY{p}{(}\PY{n}{assign\PYZus{}x\PYZus{}coord}\PY{p}{,} \PY{n}{axis} \PY{o}{=} \PY{l+m+mi}{1}\PY{p}{)}
        \PY{n}{stanton\PYZus{}strike\PYZus{}hr}\PY{p}{[}\PY{l+s+s1}{\PYZsq{}}\PY{l+s+s1}{zone\PYZus{}y}\PY{l+s+s1}{\PYZsq{}}\PY{p}{]} \PY{o}{=} \PY{n}{stanton\PYZus{}strike\PYZus{}hr}\PY{o}{.}\PY{n}{apply}\PY{p}{(}\PY{n}{assign\PYZus{}y\PYZus{}coord}\PY{p}{,} \PY{n}{axis} \PY{o}{=} \PY{l+m+mi}{1}\PY{p}{)}
        
        \PY{c+c1}{\PYZsh{} Plot Stanton\PYZsq{}s home run zone as a 2D histogram with a colorbar}
        \PY{n}{plt}\PY{o}{.}\PY{n}{hist2d}\PY{p}{(}\PY{n}{stanton\PYZus{}strike\PYZus{}hr}\PY{p}{[}\PY{l+s+s1}{\PYZsq{}}\PY{l+s+s1}{zone\PYZus{}x}\PY{l+s+s1}{\PYZsq{}}\PY{p}{]}\PY{p}{,} \PY{n}{stanton\PYZus{}strike\PYZus{}hr}\PY{p}{[}\PY{l+s+s1}{\PYZsq{}}\PY{l+s+s1}{zone\PYZus{}y}\PY{l+s+s1}{\PYZsq{}}\PY{p}{]}\PY{p}{,} \PY{n}{bins} \PY{o}{=} \PY{l+m+mi}{3}\PY{p}{,} \PY{n}{cmap}\PY{o}{=}\PY{l+s+s1}{\PYZsq{}}\PY{l+s+s1}{Blues}\PY{l+s+s1}{\PYZsq{}}\PY{p}{)}
        \PY{n}{plt}\PY{o}{.}\PY{n}{title}\PY{p}{(}\PY{l+s+s1}{\PYZsq{}}\PY{l+s+s1}{Giancarlo Stanton Home Runs on}\PY{l+s+se}{\PYZbs{}n}\PY{l+s+s1}{ Pitches in the Strike Zone, 2015\PYZhy{}2017}\PY{l+s+s1}{\PYZsq{}}\PY{p}{)}
        \PY{n}{plt}\PY{o}{.}\PY{n}{gca}\PY{p}{(}\PY{p}{)}\PY{o}{.}\PY{n}{get\PYZus{}xaxis}\PY{p}{(}\PY{p}{)}\PY{o}{.}\PY{n}{set\PYZus{}visible}\PY{p}{(}\PY{k+kc}{False}\PY{p}{)}
        \PY{n}{plt}\PY{o}{.}\PY{n}{gca}\PY{p}{(}\PY{p}{)}\PY{o}{.}\PY{n}{get\PYZus{}yaxis}\PY{p}{(}\PY{p}{)}\PY{o}{.}\PY{n}{set\PYZus{}visible}\PY{p}{(}\PY{k+kc}{False}\PY{p}{)}
        \PY{n}{cb} \PY{o}{=} \PY{n}{plt}\PY{o}{.}\PY{n}{colorbar}\PY{p}{(}\PY{p}{)}
        \PY{n}{cb}\PY{o}{.}\PY{n}{set\PYZus{}label}\PY{p}{(}\PY{l+s+s1}{\PYZsq{}}\PY{l+s+s1}{Counts in Bin}\PY{l+s+s1}{\PYZsq{}}\PY{p}{)}
\end{Verbatim}


    \begin{center}
    \adjustimage{max size={0.9\linewidth}{0.9\paperheight}}{output_17_0.png}
    \end{center}
    { \hspace*{\fill} \\}
    
    \hypertarget{should-opposing-pitchers-be-scared}{%
\subsection{10. Should opposing pitchers be
scared?}\label{should-opposing-pitchers-be-scared}}

A few takeaways:

Stanton does not hit many home runs on pitches in the upper third of the
strike zone.

Like pretty much every hitter ever, both players love pitches in the
horizontal and vertical middle of the plate.

Judge's least favorite home run pitch appears to be high-away while
Stanton's appears to be low-away.

If we were to describe Stanton's home run zone, it'd be middle-inside.
Judge's home run zone is much more spread out.

The grand takeaway from this whole exercise: Aaron Judge and Giancarlo
Stanton are not identical despite their superficial similarities. In
terms of home runs, their launch profiles, as well as their pitch speed
and location preferences, are different.

Should opposing pitchers still be scared?

    \begin{Verbatim}[commandchars=\\\{\}]
{\color{incolor}In [{\color{incolor}10}]:} \PY{c+c1}{\PYZsh{} Should opposing pitchers be wary of Aaron Judge and Giancarlo Stanton}
         \PY{n}{should\PYZus{}pitchers\PYZus{}be\PYZus{}scared} \PY{o}{=} \PY{k+kc}{True}
\end{Verbatim}



    % Add a bibliography block to the postdoc
    
    
    
    \end{document}
